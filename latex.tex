\documentclass{rjthesis}
\usepackage{booktabs}        % 设置可以使用三线表
\usepackage{amsmath}         % 设置可以引用数学公式
\usepackage{graphicx}

\rjhead{王旭:基于RFID的室内定位方法}

\rjtitle{基于RFID的室内定位方法\textsuperscript{*}}
\rjauthor{王\hspace{1em}旭\textsuperscript{1}}
\rjinfor{\textsuperscript{1}(南京大学 计算机学院,江苏 南京 )\\
	王旭, E-mail: 221220034@nju.edu.cn
}

\begin{document}
	
	\rjmaketitle
	\begin{rjabstract}
		如今,基于精确位置的服务,包括识别物体或一个人的位置的服务,在生活、生产和商业活动中有着许多用途,尤其是大型仓库和大型商场。传统的GPS等定位技术可以在室外提供高质量的定位服务,但是由于四周遮挡物和各种室内影响无法在室内提供精确定位服务。室内定位技术有多种,如WiFi、蓝牙、和RFID等。其中RFID技术使得低成本精准室内定位系统的建立成为可能。LANDMARC算法是一种基于RFID的室内定位算法,引入了参考标签的概念,减少了RFID读写器的部署数量,为实际应用提供了很大可能。本文将探讨基于RFID的室内定位方法LANDMARC算法及其演变的集中算法,包括3-D定位算法和各种LANDMARC演化算法。
	\end{rjabstract}
	\rjkeywords{RFID;室内定位;LANDMARC;3-D定位}
	
	% 开始正文
    \section{引言}
    从本世纪初,物联网的概念逐渐兴起,并且与近些年逐渐变为现实,给我们的生活带来巨大的便利。通过物联网技术,我们可以把传感器、物品、人、机器等连接起来,共享信息,进行远程管理、控制与服务。现在物体物理位置感知仍是物联网应用中的重要一环,许多程序应用都需要知道对象的物理位置。多些年来,许多系统算法都解决了自动位置感知的问题。三角测量和场景分析自动位置感知的主要技术。并且最著名的定位系统是GPS、北斗等系统,但由于其依赖卫星实现位置定位,因此在确定建筑物内物体的位置上存在固有难以解决的问题。
    
    为了实现室内定位技术,业内已经开发出多项不同技术,如WiFi、UWB、蓝牙、红外技术和射频识别(RFID)等。其中RFID技术有着相当明显的优势。RFID系统主要由两个组件组成,应答器(标签)和检测器(或读取器)。RFID标签分为有源和无源两种。无源标签拥有一个耦合元件和一个小且便宜的电子微芯片,该芯片不能修改。无源标签没有内置的电池,且不需外部单独电源来运行。无源标签在读取器的响应范围内才会被激活,读取器通过耦合单元向标签提供所需的电力。有源标签则有一个内部电源,其由内部电源供电,且其数据一般可重写。有源标签可以主动发射无线电信号或定期唤醒应答,不依赖读写器功能,相对于无源标签,通信距离更远,通常可达30米以上,且可靠性高,但是其标签尺寸更大,价格更高,寿命比无源标签较短。RFID的标签的可靠性否较高,不易受环境影响,这也决定着其在室内定位技术的重要性。RFID系统主要依赖其标签和应答器和无线电信号,其标签可以嵌入到物体中并部署在恶劣的环境,对视线无要求,有着非视线性质。
    
    该技术的无接触性和非视线性质是所有RFID系统的固有优势,此外其成本低、重量轻、速度快、可识别多个标签的有点更使其被选择为室内无线定位技术的一大优势。但是传统的RFID定位检测技术仅仅依靠的是标签与读取器的信号相应。传统的方法是部署已知位置的标签或者读取器来确定目标对象的位置。如,读取器可以部署到仓库的某些外置,当带有RFID标签的物体经过读取器周围时,系统会检测到物体的大致位置,即以读取器为中心,最大半径等于响应范围的圆圈内,或者部署大量标签,带有读取器的物体经过时会相应,确定物体的大概位置。这种传统定位方式提供的位置粒度过粗,无发满足室内定位的精确要求。

    在阅读“LANDMARC: Indoor Location Sensing Using Active RFID*”,“An Indoor Localization Mechanism Using Active RFID Tag”,“RFID-Based 3-D Positioning Schemes”等论文后,本文接下来将介绍引入参考标签概念的LANDMARC算法及其优化后的一些算法和基于RFID的3-D定位算法。本文将在第二节介绍基础的LANDMARC算法及其一些问题,在第三节介绍LANDMARC算法的一些相关优化,最后会介绍在第四节介绍基于RFID的3-D定位算法,最后总结以上算法,并给出我个人的思考。

    \section{LANDMARC算法}
    LANDMARC算法是2003年密歇根州立大学和香港科技大学的科研组共同提出的一种典型的基于参考标签的RFID室内定位方法。LANDMARC算法通过引入参考标签进行辅助定位,在定位空间
    中设置参考标签矩阵,测量未知标签和参考标签的RSSI值,采用最近邻K值算法进行估算定位。LANDMARC算法算法首次引进了参考标签的思想,使得RFID用于精确室内定位成为了可能,并且能够在不放置更多的昂贵读取器的情况下提高精确性,虽然标签的数量可能会增多,但相对于减少的昂贵的读取器,成本降低许多。

    \subsection{LANDMARC的优势}  
    LANDMARC算法是一种典型的场景分析算法,属于一种主动标定算法(这会在后面第四节介绍),引入参考标签作为参考点。LANDMARC算法主要由三个优势:
    \begin{itemize}
        \item 成本低廉,使用大量便宜的RFID标签取代大量的昂贵的读取器。
        \item 较容易适应动态环境,该方法有助于抵消检测范围变化的许多环境因素,因为参考标签和定位标签在环境中收到相同的影响,所以可以从参考标签的相关信息动态更新查找定位标签的位置信息。
        \item 位置信息相对地较为准确。不过这非常依赖于参考标签数量以及读取器、标签的相对拓扑位置。
    \end{itemize}

    \subsection{LANDMARC具体机制}    
    LANDMARC算法的主要重点是通过比较定位标签和参考标签的信号强度,找出定位标签在欧氏距离上最近的k个邻居标签,并将信号强度与定位标签接近的赋上更大的权重值,然后按分权计算各个选择的邻居参考标签的位置的加权和来获得定位标签的实际位置。

    假设有一个室内定位系统,它有$n$个RFID读取器,$m$个RFID参考标签和$u$个移动标签(需要判定物理位置),如图\ref{fig:landmarc_layout},其中的$n$个RFID读取器配置为连续模式,即连续报告指定范围内的标签,在最初的论文工作中,RFID读取器的检测范围为1-8,即读取器将扫描范围从1到8,并以每个范围30秒的速率重复循环。

    把移动标签的信号强度向量定义为$\bar{S}=(S_1,S_2,...,S_n)$,其中$_i$
    
    \begin{figure}[htb]
	    \centering
	    \includegraphics[width=0.5\textwidth]{pictures/1.png}
	    \caption{LANDMARC系统中各部分摆放示意图}
	    \label{fig:landmarc_layout}
    \end{figure}
    

\end{document}