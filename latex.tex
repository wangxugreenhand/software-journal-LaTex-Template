\documentclass{rjthesis}
\usepackage{booktabs}        % 设置可以使用三线表
\usepackage{amsmath}         % 设置可以引用数学公式
\usepackage{graphicx}
\usepackage{bm}
\usepackage{gbt7714}

\usepackage{url}

\rjhead{王旭:基于RFID的室内定位方法}

\rjtitle{基于RFID的室内定位方法\textsuperscript{*}}
\rjauthor{王\hspace{1em}旭\textsuperscript{1}}
\rjinfor{\textsuperscript{1}(南京大学 计算机学院,江苏 南京 )\\
	王旭, E-mail: 221220034@nju.edu.cn
}

\begin{document}
	
	\rjmaketitle
	\begin{rjabstract}
		如今,基于精确位置的服务,包括识别物体或一个人的位置的服务,在生活、生产和商业活动中有着许多用途,尤其是大型仓库和大型商场。传统的GPS等定位技术可以在室外提供高质量的定位服务,但是由于四周遮挡物和各种室内影响无法在室内提供精确定位服务。室内定位技术有多种,如WiFi、蓝牙、和RFID等。其中RFID技术使得低成本精准室内定位系统的建立成为可能。LANDMARC算法是一种基于RFID的室内定位算法,引入了参考标签的概念,减少了RFID读写器的部署数量,为实际应用提供了很大可能。本文将探讨基于RFID的室内定位方法LANDMARC算法及其演变的集中算法,包括3-D定位算法和各种LANDMARC演化算法。
	\end{rjabstract}
	\rjkeywords{RFID;室内定位;LANDMARC;3-D定位}
	
	% 开始正文
    \section{引言}
    从本世纪初,物联网的概念逐渐兴起,并且与近些年逐渐变为现实,给我们的生活带来巨大的便利。通过物联网技术,我们可以把传感器、物品、人、机器等连接起来,共享信息,进行远程管理、控制与服务。现在物体物理位置感知仍是物联网应用中的重要一环,许多程序应用都需要知道对象的物理位置。多年来,许多系统算法都解决了自动位置感知的问题。三角测量和场景分析自动位置感知的主要技术。并且最著名的定位系统是GPS、北斗等系统,但由于其依赖卫星实现位置定位,因此在确定建筑物内物体的位置上存在固有难以解决的问题。
    
    为了实现室内定位技术,业内已经开发出多项不同技术,如WiFi、UWB、蓝牙、红外技术和射频识别(RFID)等。其中RFID技术有着相当明显的优势。RFID系统主要由两个组件组成,应答器(标签)和检测器(或读取器)。RFID标签分为有源和无源两种。无源标签拥有一个耦合元件和一个小且便宜的电子微芯片,该芯片不能修改。无源标签没有内置的电池,且不需外部单独电源来运行。无源标签在读取器的响应范围内才会被激活,读取器通过耦合单元向标签提供所需的电力。有源标签则有一个内部电源,其由内部电源供电,且其数据一般可重写。有源标签可以主动发射无线电信号或定期唤醒应答,不依赖读写器功能,相对于无源标签,通信距离更远,通常可达30米以上,且可靠性高,但是其标签尺寸更大,价格更高,寿命比无源标签较短。RFID的标签的可靠性否较高,不易受环境影响,这也决定着其在室内定位技术的重要性。RFID系统主要依赖其标签和应答器和无线电信号,其标签可以嵌入到物体中并部署在恶劣的环境,对视线无要求,有着非视线性质。
    
    该技术的无接触性和非视线性质是所有RFID系统的固有优势,此外其成本低、重量轻、速度快、可识别多个标签的有点更使其被选择为室内无线定位技术的一大优势。但是传统的RFID定位检测技术仅仅依靠的是标签与读取器的信号相应。传统的方法是部署已知位置的标签或者读取器来确定目标对象的位置。如,读取器可以部署到仓库的某些外置,当带有RFID标签的物体经过读取器周围时,系统会检测到物体的大致位置,即以读取器为中心,最大半径等于响应范围的圆圈内,或者部署大量标签,带有读取器的物体经过时会相应,确定物体的大概位置。这种传统定位方式提供的位置粒度过粗,无法满足室内定位的精确要求。

    在阅读“LANDMARC: Indoor Location Sensing Using Active RFID*”\cite{ni2003landmarc},“An Indoor Localization Mechanism Using Active RFID Tag”\cite{jin2006indoor},“RFID-Based 3-D Positioning Schemes”\cite{wang2007rfid}等论文后,本文接下来将介绍引入参考标签概念的LANDMARC算法及其优化后的一些算法和基于RFID的3-D定位算法。本文将在第二节介绍基础的LANDMARC算法及其一些问题,在第三节介绍LANDMARC算法的一些相关优化,最后会介绍在第四节介绍基于RFID的3-D定位算法,最后总结以上算法,并给出我个人的思考。

    \section{LANDMARC算法}
    LANDMARC算法\cite{ni2003landmarc}是2003年密歇根州立大学和香港科技大学的科研组共同提出的一种典型的基于参考标签的RFID室内定位方法。LANDMARC算法通过引入参考标签进行辅助定位,在定位空间
    中设置参考标签矩阵,测量未知标签和参考标签的RSSI值,采用最近邻K值算法进行估算定位。LANDMARC算法算法首次引进了参考标签的思想,使得RFID用于精确室内定位成为了可能,并且能够在不放置更多的昂贵读取器的情况下提高精确性,虽然标签的数量可能会增多,但相对于减少的昂贵的读取器,成本降低许多。

    \subsection{LANDMARC的优势}  
    LANDMARC算法是一种典型的场景分析算法,属于一种主动标定算法(这会在后面第四节介绍),引入参考标签作为参考点。LANDMARC算法主要由三个优势:
    \begin{itemize}
        \item 成本低廉,使用大量便宜的RFID标签取代大量的昂贵的读取器。
        \item 较容易适应动态环境,该方法有助于抵消检测范围变化的许多环境因素,因为参考标签和定位标签在环境中受到相同的影响,所以可以从参考标签的相关信息动态更新查找定位标签的位置信息。
        \item 位置信息相对地较为准确。不过这非常依赖于参考标签数量以及读取器、标签的相对拓扑位置。
    \end{itemize}

    \subsection{LANDMARC具体机制}    
    LANDMARC算法的主要重点是通过比较定位标签和参考标签的信号强度,找出定位标签在欧氏距离上最近的k个邻居标签,并将信号强度与定位标签接近的赋上更大的权重值,然后按分权计算各个选择的邻居参考标签的位置的加权和来获得定位标签的实际位置。

    假设有一个室内定位系统,它有$n$个RFID读取器,$m$个RFID参考标签和$u$个移动标签(需要判定物理位置),如图\ref{fig:landmarc_layout},其中的$n$个RFID读取器配置为连续模式,即连续报告指定范围内的标签,在最初的论文工作中,RFID读取器的检测范围为1-8,即读取器将扫描范围从1到8,并以每个范围30秒的速率重复循环。

    \begin{figure}[htb]
	    \centering
	    \includegraphics[width=0.5\textwidth]{pictures/1.png}
	    \caption{LANDMARC系统中各部分摆放示意图}
	    \label{fig:landmarc_layout}
    \end{figure}

    把一个移动标签的信号强度向量定义为$\bar{S}=(S_1,S_2,...,S_n)$,其中$_i$表示的是第$i$个RFID读取器接收到的该移动标签的信号强度,$i \in{(1,n)}$。每个移动标签都有自己的信号强度向量。对于每一个参考标签,把其对应的信号强度向量标记为$\theta = (\theta_1,\theta_2,...,\theta_n)$,其中$\theta_i$表示第i个RFID读取器接收到的该参考标签的信号强度。

    如前面所介绍,在信号强度中引入欧氏距离,即利用RSSI值计算欧氏距离。对于每一个移动标签$P$,其中$P\in(1,u)$,计算其与每一个参考标签在信号强度上的欧氏距离$E_j=\sqrt{\sum_{i=1}^{n}(\theta_i-S_i)^2} $,因为信号强度反映着每一个标签相对于每一个读取器的相对距离。不妨用$E$表示一个移动标签与一个参考标签的距离,其中E的值越小,越表示该移动标签与该参考标签距离“越近”。当有m个参考标签时,则有距离向量$\bar{E}=(E_1,E_2,...,E_m)$。下面将使用这个距离向量去使用最近k-邻算法找到物理上距离移动标签最近的参考标签。

    在得到距离向量的基础上,选择最小的k个,其对应的参考标签可以被认为是距离移动标签最近的邻居。其中这里的E在原方法中使用的平均功率等级,因为其读取器的功率等级是及其容易获得到,并且也反映着相对大小关系,不过这肯定比直接使用信号强度值的效果要差一些。在获得k个邻居参考标签及其与移动标签的相对距离关系后,可以通过如下加权公式估算得到移动标签的距离坐标:
    \[
        (x, y) = \sum_{i=1}^kw_i(x_i,y_i)
    \]
    其中的权重则是另一组要确定的可调参数,如何分配决定最也后的定位精度。在原始LANDMARC算法中,认为E值反映着距离的远近,E值越大,距离“越远”,其影响程度越小,公示如下:
    \[
        w_i = \frac{\frac{1}{E_i^2}}{\sum_{i=1}^k}{frac{1}{E_i^2}}
    \]
    同理,坐标计算可以扩展到三维坐标。

    \subsection{LANDMARC算法分析}
    LANDMARC算法最大的贡献是提出了参考标签的概念,使得读取器的数量可以减少,降低成本,表明RFID室内定位是一种可行的、经济有效的室内定位候选方法。但是LANDMARC算法的精确度收到如下因素影响:参考标签的数量,只有数量躲起来,参考位置才有价值,误差才能减小;参考标签与读取器的位置选取,它们的拓扑结构也会影响最后的位置精度;RFID的信号强度是有误差的,或者其平均功率等级分辨度较小;k值的选择,k值过大,可能导致加权效果趋向于平均化,过小的话,出错的概率增大,极可能出现较大误差;E值的计算与$w_i$的计算都可以优化,这些决定着最终选择的临近参考标签以及位置的计算,而在原始算法中,处理不够精细,无法达到一个更高精度的位置信息。

    \section{LANDMARC算法的优化}
     LANDMARC算法有如下两个明显问题:
     \begin{itemize}
         \item 计算冗余:对于一个移动标签,并不是所有的RFID读取器都可以检测到该移动标签,那么超出能检测到移动标签的读取器的检测范围的的参考标签就没有计算与该移动标签的相对距离向量的必要了。如图\ref{fig:landmarc_layout}中,只有读取器$2,3,4$可以检测到目标标签,读取器1检测不到目标标签,那么无法被读取器$2,3,4$检测到的参考标签没有与目标前计算相对距离的必要。但是原始LANDMARC算法中没有考虑这个问题,导致不必要的计算,可能会导致系统延迟。
         \item 移动标签的位置依赖临近的参考标签,其误差局限在所选取的k个邻居参考标签组成的多边形中,因此为了减小误差,要提高参考标签的密度,可是当部署过多时,参考标签之间的干扰会更严重,会影响到标签与读取器之间检测信号强度的准确性。
     \end{itemize}
     
    针对这个问题,IEEE2006年的文章"An Indoor Localization Mechanism Using Active RFID Tag"\cite{jin2006indoor}提出了LANDMARC算法的优化版本来解决上面提到的两个问题。该文章提出在与之前相同环境配置的情况下,有$p$哥读取器检测到移动标签,选择一个$q$值($3\leq q \leq p$),这个q根据能接受的计算速度来决定。能同时被这$q$个读取器检测到的参考标签则作为候选参考标签集合$R$,其中集合$R$的大小为$r$。获得$\bar E=(E_1,E_2,...,E_r)$,然后从这个候选集合中选取最小E值的k个邻居参考标签。

    在最后的移动标签坐标的计算中,该工作提出通过引入三角测量机制来更精确计算移动标签的坐标。在原始LANDMARC算法中,计算移动标签的位置算法过于简单,没有使用较为科学的计算坐标方式。在改进方法中,该工作认为应该通过三角测量机制,在选择的k个参考标签中,针对每一个标签,通过三角测量机制使用剩下k-1个标签计算一个坐标$(x_i^{'} , y_i^{'})$,然后计算出$(\Delta x_i,\Delta y_i)=(x,y)-(x_i^{'} , y_i^{'})$。如图\ref{fig:2}。然后得到$(\Delta x_1,\Delta y_1),(\Delta x_2,\Delta y_2),...,(\Delta x_k,\Delta y_k)$。这样最后通过三角测量机制利用这k个参考标签计算移动标签的坐标$(x^{'},y^{,})$,再加上平均误差(三角测量机制的)得到移动标签的最后坐标。这里的平均误差为:$(\Delta x, \Delta y) = (\frac{1}{k}\sum_{i=1}^{k}\Delta x_i  ,\frac{1}{k}\sum_{i=1}^{k}\Delta y_i )$。
    这样就可以尽量地较少测量机制的误差影响。

    但我个人仍认为最后的计算位置方式有一个问题:没有考虑到k个邻居参考标签的权重问题,这会导致最终的精度问题,可以使用加权最小二乘三角测量机制,将权重考虑进去,这样计算方式更为合理。

    \begin{figure}[htb]
	    \centering
	    \includegraphics[width=0.5\textwidth]{pictures/2.png}
	    \caption{实际位置与计算位置之间的误差}
	    \label{fig:2}
    \end{figure}

    \section{基于RFID的3-D定位}
    在LANDMARC算法提出后的一年内,先后出现了许多基于RFID和参考标签概念的室内定位方法,其中一种就是IEEE 2007年"RFID-Based 3-D Positioning Schemes"\cite{wang2007rfid}提出的基于RFID的3-D定位系统。
    
    该系统的场景是一个封闭六面体空间,如仓库、集装箱等,将一些已知位置的参考标签或者参考读取器部署在密闭空间内作为参考点,以便确定放置在六面体内并附有标签后读取器的目标物体的三维坐标,且该工作的目标是将坐标误差限制在六面体最长边缘的5\%以内,以满足集装箱等密闭空间的三维包装和跟踪等精确位置要求。

    3-D定位系统中提出两种定位方案:主动方案和被动方案。主动方案的目标是定位RFID读取器,其部署RFID标签;被动方案的目标是定位RFID标签,该标签附着在目标物体上,其部署读取器在六面体中。

    \begin{figure}[htb]
	    \centering
	    \includegraphics[width=0.5\textwidth]{pictures/3.png}
	    \caption{主动方案}
	    \label{fig:3}
    \end{figure}

    \subsection{主动方案}
    主动方案旨在定位附着有RFID读取器的目标物体。场景如下:在六面体的地板和天花板上放置若干坐标已知的参考标签,如图\ref{fig:3}。不妨设六面体的长、宽、高为L,W,H。在较大的六面体中,优先选择有源标签,因为其传输范围更长,若体积不是太大,无源标签也是可以的。可以假设这些参考标签的位置坐标经过微调后是精确的,并将RFID读取器放在六面体内部,其有着固定的功率,该方案没有必要根据接受到的信号强度来估计距离,这也能减小误差。以上的参考标签的位置信息可以存储在数据服务器,也可以存储在标签中。

    为了计算六面体内部的读取器的位置$(x,y,z)$,要激活该读取器传输范围内的标签并获取其相应。如图\ref{fig:3},该读取器的信号传输范围是一个球体,该球体在地板和天花板上分别形成一个圆形,圆心为$(x_f,y_f,0),(x_c,y_c,H)$,半径分别为$r_f,r_c$。下面就要确定圆心和半径来确定球体的中心位置也就是读取器的位置。但是这是不容易的,尤其当读取器在六面体的边界时,会产生部分圆,主动方案提出了三种计算方式:

    \textbf{内边界法}:确定两个圆的“内边界”点,然后通过非线性优化确定其半径和圆心。具体如下:“内边界”点是指被激活的参考标签,其有未被激活的参考标签邻居。不妨设在地板上有$N_f$个“内边界”点,形成一个有效参考标签集合:$\Psi = \{(x_i^{f},y_i^{f},0)|1\leq i\leq N_f\}$。然后使用单纯形法计算出地板上圆形的圆心和半径。具体如下:定义一个误差目标函数:
    \[
        \varepsilon_f = \sum_{i=1}^{N_f}(r_f - \hat{L_i^{f}}^2)
    \]
    其中
    \[
        \hat{L_i^{f}} = \sqrt{(x_i^{f} - x_f)^{2} + (y_i^{f} - y_f)^{2}}
    \]
    然后采用Nelder-Mead单纯形法去最小化目标误差函数,到的$(x_f,y_f,0)$与$r_f$。然后使用相同方法求出天花板的圆心和半径,则读取器的$x$和$y$坐标有两个圆心的中心得到,$z$坐标则由$r_f$和$r_c$和$H$得到:
    \[
        z = \frac{r_c^2 - r_f^2 + H^2}{2H}
    \]
    如此就获得了读取器的坐标位置。

    \textbf{全边界法}:此法与内边界法一致,只是$\Psi$改变了,不仅仅包括“内边界”,也包括“外边界”点,即那些未被激活但是有被激活的邻居标签的参考标签。后续的计算流程与上述方法一致,不过参考标签增加了,有望提高$r_c$和$r_f$的准确性。

    \textbf{实心圆法}:此法的$\Psi$则是包括了所有别激活的参考标签,当形成一个完整的圆形时,有望提高准确性,不过当读取器在六面体的角落,激活的标签形成部分圆时,可能就会产生较大的误差。

    \subsection{被动方案}
    被动方案则是部署若干读取器和RFID参考标签在六面体中,旨在定位附着在目标物体的参考标签。除了在地板和天花板上部署参考标签外,还部署N$(4 \leq N)$个读取器在天花板上,如图\ref{fig:4}。此方案与LANDMARC算法一样要求读取器具有多个传输功率等级,这里假设读取器具有k个传输功率等级,,且能够随着功率的增加线性增加响应范围。

    \begin{figure}[htb]
	    \centering
	    \includegraphics[width=0.5\textwidth]{pictures/4.png}
	    \caption{被动方案}
	    \label{fig:4}
    \end{figure}

    如下是具体的求解算法:当接受到定位一个或多个RFID标签的请求是,所有读取器从最低功率开始,逐渐增大发射功率,直到接受到目标标签的响应,同时,每个读取器接收到参考标签的响应。不妨令$\Psi_k$表示当读取器功率等级为k时能被激活的标签集合,考虑读取器$i$和目标标签$j$,如果$j \notin \Psi_k$、$j \in \Psi_{k+1}$,那么该读取器与该目标标签的距离$L_{ij}$可以通过计算该读取器到其他属于$\Psi_{k+1}$,不属于$\Psi_k$的参考标签的距离的平均值获得。如此计算出目标标签$j$到$N$个读取器的距离$L_{1j},L_{2,j},...,L_{N,j}$。然后与主动方案类似,通过非线性最优化求得目标标签的位置坐标。即最小化如下目标误差函数:
    \[
        \varepsilon = \sum_{i=1}^{N}(\frac{L_{ij}-\hat{L_{ij}}}{L_{i,j}})^2
    \]
    其中,设$(x_i,y_i,z_i)$和$(x_j,y_j,z_j)$分别是读取器i和目标标签j的坐标,那么二者的计算距离$\hat{L_{ij}}$计算公式如下:
    \[
        \hat{L_{ij}} = \sqrt{(x_i - x_j)^2 + (y_i - y_j)^2 + (z_i - z_j)^2}
    \]
    当然,如果有某一个读取器调到最大功率也未能检测到目标标签,则从上述目标误差式子中去掉对应一项。最后,使用与主动方案中一致的单纯形法解出目标标签的最优位置坐标。
    这里没有和主动方案一致地讨论边界的问题,因为$\Psi_{k+1}$中激活而$\Psi_k$中未激活的标签已经形成了一个圆环区域。

    \subsection{算法分析}
    两种方案都继承了参考标签的想法,并且引入非线性最优化算法来求解最后的目标坐标,这比LANDMARC算法要精确一些。主动方案中不需要拥有多种发射功率的读取器,而且读取器的数量较少,而被动方案就需要一批具有多种发射功率的读取器,成本高一些。与被动方案相比,主动方案的硬件成本更低,大多情况下的精度更高一些,导致更小的误差;另一方面,被动方案在同时定位多个目标物体的位置比主动方案更有效。

    \section{总结}
    基于RFID的低成本高精度室内定位技术是可能的,并且应用价值是较大的。在LANDMARC算法中“参考标签”概念的基础上,可以使用更精确的优化算法、更精确的硬件来提高该系列算法的精确性。但RFID固有的一些问题也是进一步增加精确度的障碍,如RFID的信号强度有可能收到环境的影响,以及彼此之间的干扰等问题。LANDMARC算法在信号接收、彼此之间的距离估算、邻居选取、权重的选取都有一些瑕疵,都可以进一步研究。
    后续也有一些工作在LANDMARC算法的基础上加上更细粒度的过滤方式(如高斯过滤),使用机器学习算法如支持向量机\cite{xu2018rfid}选取邻居等方式增加LANDMARC算法的精确度和稳定性,这也是基于RFID的室内定位算法的一种进一步演进工作。

    
\bibliographystyle{gbt7714-numerical}
\bibliography{refs} 
    

    
\end{document}